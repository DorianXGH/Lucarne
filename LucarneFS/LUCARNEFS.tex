\documentclass{article}
\usepackage[british]{babel}
\usepackage[T1]{fontenc}
\usepackage{times}

\title{LucarneFS Specification}
\author{Dorian Bourgeoisat}
\date{\today\\v0.1}

\begin{document}
\maketitle
\newpage
\tableofcontents

\newpage
\section{Definitions}
\subsection{Chain}
Linked list with the last link pointing to the first.
\subsection{First/Last link of a chain}
As a result of the structure of a chain, the last or first links are defined by the objects that hold the structure. The link pointed to by the holding structure is the first link. The link pointing to it is the last link.

\section{General Principles}
\subsection{Chain of blocks}
The underlying structures are almost always a chain. The exception to this is the block itself, or a contiguous file. A block is always 512 bytes long.\newline
The pointer to the next block is a 6 bytes long relative LBA 48 adress little-endian at the end of the block. It has a duplicate. Between the 2 pointers is a 2 bytes long volume identifier and a 2 bytes long metadata bitfield.

\begin{tabular}{|c|c|}
    \hline
    Rel. LBA & Content \\
    \hline
    ... & ...  \\
    \hline
    496 & NEXT\_BLOCK\_LBA48\_DUPLICATE \\
    \hline
    502 & METADATA  \\
    \hline
    504 & VOLUME\_ID  \\
    \hline
    506 & NEXT\_BLOCK\_LBA48  \\
    \hline
\end{tabular}
\newline
This 16 bytes long structure will be called in the following "block pointer".
\subsection{Free blocks}
Free blocks are maintained in a chain, when a file is deleted, its blocks are put in the beginning of the chain. When a file must be given new blocks, they are taken from the beginning of the chain.
\subsection{Writing to blocks} 
To write data in a file, new blocks are allocated and wrtten to with the last block allocated pointing to the rest of the file. A special block is used to keep track of the first replaced block.
Then, the previous block of the file will be updated to point to the new blocks.
The replaced blocks will be put in the free blocks chain using the same method.

\subsection{Hierarchy}
Each folder and file is described in a block in a chain. The content of a folder/file is the chain a second block pointer points to in the folder/file block.
\newline
\begin{tabular}{|c|c|}
  \hline
  Rel. LBA & Content \\
  \hline
  ... & ...  \\
  \hline
  480 & BLOCK\_POINTER\_CONTENT \\
  \hline
  496 & BLOCK\_POINTER\_NEXT \\
  \hline
\end{tabular}

\subsection{Volume identifier}

\begin{tabular}{|c|c|c|}
  \hline
  4 first bits of drive GUID & first byte of partition GUID & 4 bits collision prevention \\
  \hline
\end{tabular}
\newline
The 4 bits collision prevention is generated at partitionning. If there is a collision at partitioning with another partition, the prevention number is incremented until no collisions are found. If this fails to resolve collisions, increment the first byte of partition GUID until no collisions found. If this also fails to resolve collisions, increment 4 first bits of drive GUID.
\newline
Note : This operation can be interpreted and implemented as an increment to the 16 bits uint interpretation of the volume identifier.
\section{Block Types}
\subsection{Master}
The master block is the first block of the partition. It contains a pointer to the root folder. Along with the volume identifier of the partition, and filesystem structures.
\newline
\begin{tabular}{|c|c|}
  \hline
  Rel. LBA & Content \\
  \hline
  0 & ASCII for "LuFS"  \\
  \hline
  4 &  FS\_VERSION \\
  \hline
  5 &  FLAGS \\
  \hline
  6 &  VOLUME\_ID \\
  \hline
  8 &  SLAVE\_VOLUMES[64] \\
  \hline
  136 & RESERVED \\
  \hline
  464 & BP\_CURRENT\_OP \\
  \hline
  480 & BP\_FREE \\
  \hline
  496 & BP\_ROOT \\
  \hline
  
\end{tabular}
\newline
Flags :
\begin{tabular}{|c|c|c|c|c|c|c|c|}
  \hline
  7 & 6 & 5 & 4 & 3 & 2 & 1 & 0 \\
  \hline
  RES & RES & RES & RES & RES & RES & RES & SLAVE  \\
  \hline
\end{tabular}
\subsection{Folder}
\subsection{File}
\subsection{Data}
\end{document}